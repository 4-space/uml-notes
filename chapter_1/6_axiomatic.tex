\section{Kleinberg's Impossibility}
\begin{enumerate}
\item \textbf{Scale-Invariance}

  Choice of unit should not affect clustering
  $$f(x,d)=f(x,\alpha\cdot d),\ \text{for any}\ \alpha>0$$
\item \textbf{Richness}

  Different $d$'s can give different partitions. All partitions are
  possible by changing $d$.  
\item \textbf{Consistency}

  Reducing distance within clusters and increasing distance between
  clusters won't change partition. 
  \begin{align*}
    \text{Suppose}\ &f(x,d)\ \textrm{gives a partition}\\ 
    \bar{d}(i,j)&\leq
    d(i,j)\ \text{for}\ i,j\ \textrm{in the same cluster}\\ 
    \bar{d}(i,j)&\geq d(i,j)\ \textrm{for $i,j$ in different clusters}\\
    f(x,\bar{d})&=f(x,d)
  \end{align*}
\end{enumerate}
\begin{theorem}
  There exists no $f$ which satisfies axiom $1,2~\&~3$.
\end{theorem}

\begin{proof}
  Suppose there is a set of three points$\{x_1, x_2,x_3\}$. Two
  distance function $d$ and $d'$ such that $f(x,d)$ gives a clustering
  of $\{\{x_1\},\{x_2\},\{x_3\}\}$ and $f(x,d')$ gives a clustering of
  $\{\{x_1,x_2\},\{x_3\}\}$.
  
  It can be observed that $$f(x,d)\neq f(x,d')\hspace{5em}(1)$$
  
  By scale-invariance, $$f(x,\alpha\cdot d') =
  f(x,d')\hspace{5em}(2)$$
  
  We can find an $\alpha$ that $\alpha\cdot d'$ enlarges distance
  between any two points. If consistency holds, it means new distance
  function $\alpha\cdot d'$ shouldn't change partition result of
  $f(x,d)$ because $\alpha\cdot d'$ increases all between-cluster
  distances. However, from (1) and (2) we know that $f(x,d)\neq
  f(x,\alpha\cdot d')$, so consistency doesn't hold for the partition
  function $f$.
\end{proof}
Note that k-means algorithm is not richness because it can only have
$k$ clusters. 
