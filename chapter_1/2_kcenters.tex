\section{The \emph{k}-centers problem}
Consider the following optimization problem on an arbitrary metric
space:
\begin{enumerate}
\item \underline{Input}: a set of $n$ points $x_1,....x_n \in 
\mathcal{X}$ assuming $(X,\rho)$; a positive integer $k$ 
\item \underline{Output}: $T \subset \mathcal{X}$ s.t. $|T| = k$
\item \underline{Goal}: minimize the cost of $T$ where: $ cost(T)
:= \max_{i \in \{1...n\}} \rho(x_i,T) $
\end{enumerate}

One possible solution is an exhaustive search, which has exponential
time complexity. In fact, this problem is NP-hard, so we might look
instead for an approximate solution. The farthest first traversal
algorithm, which has polynomial complexity, does this.

\begin{algorithm}
\caption{Farthest First Traversal Algorithm (FFTA)}
\begin{algorithmic} 
\STATE Pick randomly $z\in \mathcal{X}$ and set $T = \{z\}$\;
\WHILE{$|T| < k$}
\STATE $z = \argmax_{x\in \mathcal{X}} \rho(x,T)$ \;
\STATE $T = T \cup \{z\}$\;
\ENDWHILE
\end{algorithmic}
\end{algorithm}

\begin{example}
To see that this is not an optimal solution to the k-center problem,
consider the following counterexample. Define the 2-center problem
($|T| = 2$) on the space $\mathcal{X} = \{0, \frac{1}{4}, \frac{1}{2},
\frac{3}{4}, 1\}$, with the usual metric $\rho$. The farthest fast
traversal algorithm might yield to  $\{0,1\}$ while the optimal
solution is $\{\frac{1}{4},\frac{3}{4}\}$.
\end{example}

\begin{theorem}
Let $T^*$ be the optimal solution of the k-center problem and
$\mathrm{cost}(T^*)$ be its optimal cost. Let T be a solution given
by FFTA. Then: 
\[cost(T^*) \le cost(T) \le 2 * cost(T^*)\]
Thus, FFTA is 2-optimal for the k-center problem. 
\end{theorem} 

\begin{proof}
Let $ cost(T) = r = \max_{s \in \mathcal{X}} \rho(s,T) $. There exists
$x_0 \in \mathcal{X}$ such that $x_0 = \argmax_{x \in\mathcal{X}}
\rho(x,T)$. We set $T' = T \cup \{x_0\} $, and observe that $\forall
t_i, t_j \in T', t_i \ne t_j : \rho(t_i,t_j) \ge r$, because if $x_0
\notin T$, then during any iteration of the farthest first traversal
algorithm, for the new center $t_i$ added to $T$ it must be true that
$\rho(t_i, T) \geq r$, otherwise $x_0$ would be already added to $T$.
We note that $|T'| = k+1$ and $ |T^*| = k$. Because there are $k$
elements in $T^*$ covering $k+1$ elements in $T'$, by the pigeonhole
principle, there exists some $t^* \in T^*$ that covers at least two
elements $t_1,t_2 \in T'$. As $\rho(t_1,t_2) \geq r$, by the triangle
inequality, at least one of $\rho(t_1,t^*)$ or $\rho(t_2,t^*)$ is at
least $\frac{r}{2}$. This implies $\mathrm{cost}(T^*)\geq\frac{r}{2}$.
\end{proof}

\begin{remark}
Obtaining an $(2 - \epsilon)$-approximation is NP-hard for general
metric spaces.
\end{remark}

\noindent\textbf{Some related open problems:} Hardness in Euclidean
spaces. Is k-center problem still hard in these spaces? Can we get
better than 2-optimality? Is this the better algorithm to solve the
problem? 
